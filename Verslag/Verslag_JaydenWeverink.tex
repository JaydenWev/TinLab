% Persoonlijk verslag van TinLab AA
% Mathematical Symbols:
% https://oeis.org/wiki/List_of_LaTeX_mathematical_symbols
\documentclass{article}% article report slides book letter beamer memoir minimal proc 
\usepackage{graphicx}
\usepackage{verbatim}% for multi line comments
\usepackage[dutch]{babel}
\usepackage{gensymb}% graden teken
\usepackage{multicol} %
\begin{document}
\sffamily
\begin{titlepage}
  \centering
    \vfill
    {\bfseries\Huge
      Verslag Tinlab Advanced Algorithms \\
        \vskip2cm
      }
      {\bfseries\Large
        J. I. Weverink\\
      }
      {
        \bfseries\normalsize
        \ldots\\
        \vskip1cm
        \today\\
    }    
    \vfill
    \includegraphics[width=4cm]{pictures/logohr.png} % also works with logo.pdf
    \vfill
    \vfill
\end{titlepage}
\newpage
\tableofcontents

\newpage
\section{Inleiding}
Zie hier een referentie naar Royce~\cite{royce1987managing} en nog een naar Clarke~\cite{modelchecking}\ldots 

\section{Requirements}

\subsection{Requirements}
% Wat zijn requirements?
% behoefte of noodzaak
Requirements zijn beschrijvingen over hoe een product zou moeten functioneren. Zo verandert de betekenis van een requirement als de machine in een andere omgeving wordt geplaats. De requirements voor de verwaring van een ruimte bijvoorbeeld: Binnen moet het altijd warm zijn. In nederland kunnen we zeggen dat 25$^{\circ}$C als warm wordt aangezien. Terwijl op de noordpool dat op een lager punt zal zijn.

Anders gezegd zijn requirements geen harde eisen. Dit komt doordat de requirements zijn geformuleerd van het perspectief van de opdrachtgever. De opdrachtgever kan de requirtements geven zonder kennis te hebben van de machine die het moet gaan uitvoeren. De requirements die zijn opgesteld geven dan ook geen grenzen aan die overscheden kunnen worden.

% system requirements software requirements 
Onder requirements zijn er verschillende soorten requirements. Zo zijn system requirements opgesteld voor het hele systeem en bevatten subsystemen die die kunnen bestaan uit software en hardware. Hier moet uiteindelijk alles ervoor zorgen dat deze requirement wordt gehaald. Software requirement zijn niet bedoeld voor het hele systeem, maar behappen alleen de de software van het systeem. Software requirement zijn niet bedoeld voor het hele systeem, maar behappen alleen de software van het systeem. De software requirements kunnen gaan over over de functionele eisen, gebruikers eisen en zakelijke vereisten.
%software requirements: https://books.google.nl/books?hl=nl&lr=&id=nbpCAwAAQBAJ&oi=fnd&pg=PT32&dq=what+are+software+requirements&ots=9pK_H20xTk&sig=tEhuqR8M3pOWPPM4kDu4yOSfOys&redir_esc=y#v=onepage&q=what%20are%20software%20requirements&f=false
Requirements zijn onder te verdelen in verschillende delen: 
\begin{multicols}{2}
\begin{itemize}
\item Functional Requirement
\item Performance requirement
\item Usability requirement
\item User requirement
\item Interface requirement
\item Modes requirement
\item Adaptability requirement
\item Physical requirement
\item Design requirement
\item Enviromental requirement
\item Logistical requirement
\end{itemize}
\end{multicols}
TODO SOFTWARE REQUIREMENTS\\

%	https://www.sebokwiki.org/wiki/System_Requirements#Definition_and_Purpose_of_Requirements
% Hoe krijgt men requirements
% Wat voor requirements elication technieken zijn er zoal?
% 	functional, performance, constraint
% Wat is het verschil tussen functionelen en non--functionelen requirements?
\subsubsection{Functionele eisen en niet-functionele eisen}
Functionele eisen geven aan wat het systeem moet doen en kunnen. Niet-functionele eisen geven de eigenschappen aan van het systeem, zoals snelheid, veiligheid en bruikbaarheid. Met andere woorden functionele eisen geven informatie over het "wat". niet-functionele eisen geven informatie over het "hoe"
% wat is mode confusion?
%	mode van een apparaat bijv. auto in sport en eco modus
% Wat verstaat men onder automatiseringparadox
\begin{comment}% ================================================================
Requirements zijn opgesteld tijdens het opzetten van een project. het systeem moet aan deze punten voldoen.
\\
Gaan uitsluitend over wereld fenomeen, in termen van het probleem niet techniek
\\
\end{comment}% ==================================================================

\subsection{specificaties}
%system specifications	software specifications
Uitspraak over gedeelde fenomenen of doelen die de machine moet bereiken middels de onderdelen waar die machine uit bestaat of middels de fenomenen waar de machine controle over heeft.
\\
Doorgaands preciezer, meetbaar, exact geformuleerd.
\\
Wel precies, beschrijven hoe de machine vanuit zijn mogelijkheden de doelen opgesteld in de requirements kan bereiken.

\subsection{Het vier variabelen model}
% Het 6 variable model
Sensoren Software Actuatoren Omgeving
\\
\includegraphics[width=10cm]{pictures/4_var_model.png}
\\
\cite{https://www.researchgate.net/figure/4-Variable-Model-of-Parnas-Madey_fig3_270733268}
% source https://www.researchgate.net/figure/4-Variable-Model-of-Parnas-Madey_fig3_270733268

\subsubsection{Monitored variabelen}
Wat gemeten wordt vanuit de omgeving:
\begin{itemize}
  \item Temperatuur
  \item Licht intensiteit
  \item Luchtvochtigheid
  \item Wat je allemaal met sensoren kunt meten
\end{itemize}

\subsubsection{Controlled variabelen}
Kunnen worden "bestuurd" door actuatoren:
\begin{itemize}
  \item Temperatuur
  \item Licht intensiteit
  \item Wat je allemaal kan beinvloeden
\end{itemize}

\subsubsection{Input variabelen}
De data, die staan voor de gemeten waardes vanuit de omgeving, die als input door de software worden gebruikt.

\subsubsection{Output variabelen}
De data die de software levert als output. Waar de actuatoren op moeten handelen.
% ===============================================================================
% ===============================================================================
% ===============================================================================
\subsection{Rampen}

\subsubsection{Ramp 1}
\begin{description}
\item[Beschrijving]
\item[Datum en plaats] 
\item[Oorzaak]
  %Beschrijf wat er mis ging in termen van het vier variabelen model/requirements/specificaties
\end{description}

\subsubsection{Ramp 2}
\begin{description}
\item[Beschrijving]
\item[Datum en plaats] 
\item[Oorzaak]
  %Beschrijf wat er mis ging in termen van het vier variabelen model/requirements/specificaties
\end{description}

\subsubsection{Ramp 3}
\begin{description}
\item[Beschrijving]
\item[Datum en plaats] 
\item[Oorzaak]
  %Beschrijf wat er mis ging in termen van het vier variabelen model/requirements/specificaties
\end{description}

\subsubsection{Ramp 4}
\subsubsection{Ramp 5}
\subsubsection{Ramp 6}

\section{Modellen}
Een goed model heeft een duidelijk object dat gemodelleerd moet worden, er is duidelijk \textbf{wat} er beschreven moet worden.
\\
Een goed model heeft een duidelijk doel.
-waarom modelleren we? (voor communicatie of verificatie, analyse, etc.)
\\
Een goed model is traceerbaar: elk onderdeel is te herleiden tot de onderdelen van het ëchte"systeem.
\\
Een goed model is waarheidsgetrouw: relevante onderdelebn van het model komen terug in de werkelijkheid.
\\
een goed model is eenvoudig, maar niet te eenvoudig
\\
Een goed model is uitbreidbaar en herbruikbaar: in de toekomst is het eenvoudig verder te werken met dit model en kunnen zelfs \textit{klassen} van vergelijkbare  systemen gemaakt worden
\\
Een goed model deelt geen jargon/semantiek met andere documenten en modellen.
\\\\
Richtlijnen (tegenstrijdig heden:
\\
Waarheidgetrouw vs simpelheid
duidelijheid vs. gedeeld jargon/semantiek
\subsection{De Kripke structuur}

\subsection{Soorten modellen}

\subsection{Tijd}

\subsection{Guards en invarianten}
Guards zijn voorwaarden waaraan moet worden voldaan voordat een status kan worden gemaakt.

\subsection{Deadlock}

\subsection{Zeno gedrag}

\section{Logica}

\subsection{Propositielogica}

\subsection{Predicatenlogica}

\subsection{Kwantoren}

\subsection{Dualiteiten}

\section{Computation tree logic}

\subsection{De computation tree}

\subsection{Operator: AG}
De betekenis van AG is makkelijk te onthouden A = Always, G = Globally. Dit houdt in dat het niet uit maakt waar je bent, je zal altijd van welke positie dan ook bij een gedefinieerd punt uitkomen.

\subsection{Operator: EG}
De betekenis van AG is makkelijk te onthouden E = Exists, G = Globally.

\subsection{Operator: AF}
De betekenis van AG is makkelijk te onthouden A = Always, F = Eventually.
\subsection{Operator: EF}

\subsection{Operator: AX}

\subsection{Operator: EX}

\subsection{Operator: p U q}

\subsection{Operator: p R q}

\subsection{Fairness}

\subsection{Liveness}

\newpage

\newpage
\bibliography{references}
\bibliographystyle{plain}
\end{document}


