%Persoonlijk verslag van TinLab AA
\documentclass{article}
\usepackage{graphicx} 
\usepackage[dutch]{babel}
\begin{document}
\sffamily
\begin{titlepage}
  \centering
    \vfill
    {\bfseries\Huge
      Verslag Tinlab Advanced Algorithms \\
        \vskip2cm
      }
      {\bfseries\Large
        J. I. Weverink\\
      }
      {
        \bfseries\normalsize
        176-671\\
        \vskip1cm
        \today\\
    }    
    \vfill
    \includegraphics[width=4cm]{logohr.png} % also works with logo.pdf
    \vfill
    \vfill
\end{titlepage}
\newpage
\tableofcontents

\newpage
\section{Inleiding}
Zie hier een referentie naar Royce~\cite{royce1987managing} en nog een naar Clarke~\cite{modelchecking}\ldots 

\section{Requirements}

\subsection{Requirements}
Requirements zijn punten opgesteld tijdens het opzetten van een project. 
Requirements zijn punten waar een systeems aan moet voldoen. 

\subsection{specificaties}

\subsection{Het vier variabelen model}
\subsubsection{Monitored variabelen}
\subsubsection{Controlled variabelen}
\subsubsection{Input variabelen}
\subsubsection{Output variabelen}

\subsection{Rampen}

\subsubsection{Ramp 1}
\begin{description}
\item[Beschrijving]
\item[Datum en plaats] 
\item[Oorzaak]
  %Beschrijf wat er mis ging in termen van het vier variabelen model/requirements/specificaties
\end{description}

\subsubsection{Ramp 2}
\begin{description}
\item[Beschrijving]
\item[Datum en plaats] 
\item[Oorzaak]
  %Beschrijf wat er mis ging in termen van het vier variabelen model/requirements/specificaties
\end{description}

\subsubsection{Ramp 3}
\begin{description}
\item[Beschrijving]
\item[Datum en plaats] 
\item[Oorzaak]
  %Beschrijf wat er mis ging in termen van het vier variabelen model/requirements/specificaties
\end{description}

\subsubsection{Ramp 4}
\subsubsection{Ramp 5}
\subsubsection{Ramp 6}

\section{Modellen}
Een goed model heeft een duidelijk object dat gemodelleerd moet worden, er is duidelijk \textbf{wat} er beschreven moet worden.
\\
Een goed model heeft een duidelijk doel.
-waarom modelleren we? (voor communicatie of verificatie, analyse, etc.)
\\
Een goed model is traceerbaar: elk onderdeel is te herleiden tot de onderdelen van het ëchte"systeem.
\\
Een goed model is waarheidsgetrouw: relevante onderdelebn van het model komen terug in de werkelijkheid.
\\
een goed model is eenvoudig, maar niet te eenvoudig
\\
Een goed model is uitbreidbaar en herbruikbaar: in de toekomst is het eenvoudig verder te werken met dit model en kunnen zelfs \textit{klassen} van vergelijkbare  systemen gemaakt worden
\\
Een goed model deelt geen jargon/semantiek met andere documenten en modellen.
\\\\
Richtlijnen (tegenstrijdig heden:
\\
Waarheidgetrouw vs simpelheid
duidelijheid vs. gedeeld jargon/semantiek
\subsection{De Kripke structuur}

\subsection{Soorten modellen}

\subsection{Tijd}

\subsection{Guards en invarianten}

\subsection{Deadlock}

\subsection{Zeno gedrag}

\section{Logica}

\subsection{Propositielogica}

\subsection{Predicatenlogica}

\subsection{Kwantoren}

\subsection{Dualiteiten}

\section{Computation tree logic}

\subsection{De computation tree}

\subsection{Operator: AG}

\subsection{Operator: EG}

\subsection{Operator: AF}

\subsection{Operator: EF}

\subsection{Operator: AX}

\subsection{Operator: EX}

\subsection{Operator: p U q}

\subsection{Operator: p R q}

\subsection{Fairness}

\subsection{Liveness}

\newpage

\newpage
\bibliography{references}
\bibliographystyle{plain}
\end{document}


